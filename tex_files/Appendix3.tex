\vspace{\baselineskip}\chapter*{Appendix C: \\ How to use the model}
\addcontentsline{toc}{chapter}{Appendix C: How to use the model}
This is a full guide intended to simplify the installation, configuration, and usage of the project. \par

\begin{enumerate}
	\item Anaconda/conda\par

Using Anaconda, for Windows, and conda, for Linux and macOS, is strongly recommended in order to have a cross-platform installation guide of the project. Anaconda/conda also makes use of the python virtual environments that allows to install and upgrade Python distribution packages without interfering with the behaviour of other Python applications running on the same system.\par

	\item Using conda\par

Once conda is installed you only have to create your first environment once:\par

{\fontsize{9pt}{10.8pt}\selectfont conda create --name <Environment Name> "python=3"\par}\par

After that, a new environment file system will be created under the "anaconda3/envs" folder and will store all the packages installations and updates. Each time a shell or cmd is launched, the following command must be typed to indicate anaconda to use a certain environment:\par

{\fontsize{9pt}{10.8pt}\selectfont $\#$  Linux and Mac distributions:\par}\par

{\fontsize{9pt}{10.8pt}\selectfont conda activate <Environment Name>\par}\par


\vspace{\baselineskip}
{\fontsize{9pt}{10.8pt}\selectfont $\#$  Windows:\par}\par

{\fontsize{9pt}{10.8pt}\selectfont source activate <Environment Name>\par}\par

	\item Setting up the project\par

Please check the following link to access the Github repository of the project: \href{https://github.com/alvarolop/digital\_comms/}{https://github.com/alvarolop/digital\_comms/}. The following commands can be used to download, install and configure the project:\par

{\fontsize{9pt}{10.8pt}\selectfont $\#$  1. Clone the repo\par}\par

{\fontsize{9pt}{10.8pt}\selectfont git clone \href{https://github.com/alvarolop/digital\_comms/}{https://github.com/alvarolop/digital\_comms/}\par}\par

{\fontsize{9pt}{10.8pt}\selectfont cd digital\_comms/\par}\par


\vspace{\baselineskip}
{\fontsize{9pt}{10.8pt}\selectfont $\#$  2. Target the correct input and output files in "./scripts/script\_config.ini":\par}\par


\vspace{\baselineskip}
{\fontsize{9pt}{10.8pt}\selectfont $\#$  - shapefile\_path: Path of the .shp files.\par}\par

{\fontsize{9pt}{10.8pt}\selectfont $\#$  - input\_folder: Path of the environment files.\par}\par

{\fontsize{9pt}{10.8pt}\selectfont $\#$  - output\_folder: Path where all the outputs will be stored. \par}\par


\vspace{\baselineskip}
{\fontsize{9pt}{10.8pt}\selectfont $\#$  3. Install the necessary dependencies using conda:\par}\par


\vspace{\baselineskip}
{\fontsize{9pt}{10.8pt}\selectfont conda update -n base conda\par}\par

{\fontsize{9pt}{10.8pt}\selectfont conda install -c conda-forge numpy\par}\par

{\fontsize{9pt}{10.8pt}\selectfont conda install -c conda-forge pyshp\par}\par

{\fontsize{9pt}{10.8pt}\selectfont conda install -c conda-forge imageio\par}\par


\vspace{\baselineskip}
{\fontsize{9pt}{10.8pt}\selectfont conda install -c plotly plotly\par}\par

{\fontsize{9pt}{10.8pt}\selectfont conda install -c anaconda basemap\par}\par


\vspace{\baselineskip}
{\fontsize{9pt}{10.8pt}\selectfont $\#$  4. Run this command once per machine to install the package :\par}\par

{\fontsize{9pt}{10.8pt}\selectfont cd digital\_comms/\par}\par

{\fontsize{9pt}{10.8pt}\selectfont python setup.py develop\par}\par


\vspace{\baselineskip}
{\fontsize{9pt}{10.8pt}\selectfont $\#$  5. Run this command to install the third party packages:\par}\par

{\fontsize{9pt}{10.8pt}\selectfont python setup.py install\par}\par


\vspace{\baselineskip}
	\item Running the project
\end{enumerate}\par

Use the following command to run the code:\par

{\fontsize{9pt}{10.8pt}\selectfont python .$\textbackslash$ scripts$\textbackslash$ run.py\par}\par


\vspace{\baselineskip}

\vspace{\baselineskip}