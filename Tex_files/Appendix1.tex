\vspace{\baselineskip}\chapter*{Appendix A: \\ Social, economic and environmental impact of the thesis}
\addcontentsline{toc}{chapter}{Appendix A: Social, economic and environmental impact of the thesis}
National infrastructure systems form the basis for the economy of a country. Building infrastructure is a long-term commitment that is very difficult to reverse, which means that infrastructure decisions have major implications for sustainability. Moreover, these types of investments require very significant capital that have long lead-times and lifetimes and, therefore, investments that are taking place right now will lock in patterns of development for decades to come. The ITRC project is creating a set of models to assist in the planning and design of national infrastructure.\par

NISMOD (National Infrastructure Systems MODel) is the infrastructure system of systems modelling platform of the ITRC and one of its models is MINERVA, which is in charge of modelling the communications subsystem and where this Master’s Thesis is framed.\par

Right now, complex decisions about future telecommunications networks are taken just using traditional deployment models that are isolated from other national infrastructures or constraints. This work is aimed to contribute to creating this great platform that will allow taking decisions with a much broader view.\par

From the economic point of view, this paper will contribute to a faster development of the telecom operators since they will have a better understanding of the impact of their long-term investments. Additionally, it will reduce the risks of carrying out some telecommunication services deployment in not-profitable areas, increasing the economic viability and, thus, bringing technology closer to people that live in small communities.\par

From the ethical perspective, this work will help in the just distribution of the radio spectrum, which is a limited good belonging to all since will help policy-makers to create better coverage obligations that will maximize the harnessing of it.\par

To sum up, this work also favours the general development of the society and, due to the enabling character of the telecommunications sector, it will help in making advances in virtually all technology sectors.\par

