\chapter{Abstract}
%\addcontentsline{toc}{part}{Abstract}

The extraordinary growth rates of the data demand in mobile phones along with the finite nature of radio spectrum make radio spectrum a scarce and precious resource. In consequence, regulatory bodies and the industry continuously look for new ways to increase spectrum use efficiency and networks’ capacity. In this regard, the International Telecommunication Union (ITU) proposed to switch from analogue to digital television, which reduced the spectrum needs for TV broadcasting and thus enabled to allocate part of the historically-used spectrum for TV, known as the digital dividend, to mobile communications systems.\par

Only a few years after the digital dividend (790-862 MHz) has been released, policy-makers are planning to undertake a second digital dividend to repurpose the 694-790 MHz band. To allocate this spectrum, regulators will need to design auctions that can guarantee that these frequencies are optimally used from both the social and business perspectives. Coverage obligations are the mechanism that policymakers use to force licensees to undertake infrastructure investments in the short run. This has been an extensively used legal mechanism to ensure certain social benefits from the private use of radio spectrum.\par

This Master Thesis contributes to the project ITRC- MISTRAL, a research consortium led by the University of Oxford (UK), that is exploring ways to improve the performance of infrastructure systems. In particular, this work focuses on the model MINERVA, which is a 5G model aiming to help policy-makers and telecom operators to assess the long-term benefits of their decisions.\par

The first objective of this work is the characterization of coverage obligations that policy-makers have forced telecom operators to fulfil and which were the success and challenges that they had in the development of the telecommunications infrastructure in the biggest European economies, particularly Spain, the UK, Germany and France. Secondly, this work has aimed to analyse how different coverage obligations would play out in the coming auction of the 700 MHz band. This band will be relevant for the rollout of future 5G telecommunication services. The subsequent analysis is focused on analysing the interest that operators would have in using this frequency band rather than higher frequencies bands \par

The contributions to the project revolve around three main issues. First, the creation of a new capacity-expansion strategy, the «700 MHz densification», to test which is the maximum capacity that this frequency band could provide. Second, the creation of four coverage obligation options, which are defined based on the main characteristics of the obligations imposed by Spain, the UK, Germany and France. Third, the creation of a visualization module that presents the desired results of the model in several forms (Graphs, maps, csv files, etc.) to help users understand the simulation results and therefore the advantages and disadvantages of each coverage obligation alternative.\par

Finally, these contributions have been used to analyse the impact of the capacity-expansion strategies and the coverage obligations. First, the newly created «700 MHz densification» has been compared with the rest of strategies to determine its viability and suitability. Second, this work presents a study of the implications of the recently defined coverage obligations.\par

\section*{Keywords}
5G mobile networks, 700MHz band, coverage obligations, second digital dividend, spectrum auctions. 





\chapter{Resumen del trabajo}
%\addcontentsline{toc}{part}{Resumen del trabajo}

El extraordinario crecimiento de la demanda de servicios móviles, combinado con la naturaleza limitada del espectro radio, convierten a éste en un recurso cada vez más escaso y preciado. Esta situación obliga a que organismos reguladores y compañías de la industria busquen continuamente formas de incrementar la eficiencia en el uso del espectro y, por tanto, en la capacidad de la red. En esta línea, la International Telecommunication Union (ITU) propuso el cambio de televisión analógica a digital, lo que redujo la cantidad de espectro necesario para servicios de radiodifusión y permitió destinar parte de este espectro a servicios de comunicaciones móviles. Esto es el llamado primer dividendo digital.\par

Pocos años después, los reguladores están planificando un segundo dividendo digital relacionado con la banda de los 700 MHz y con subastas de espectro que garanticen que la reasignación maximiza el interés social y de las empresas involucradas. Las obligaciones de cobertura son el mecanismo legal utilizado para fomentar la obtención de beneficios sociales del uso privado del espectro obligando a que los operadores que obtengan licencias lleven a cabo inversiones en un corto plazo.\par

Este Trabajo de Fin de Máster contribuye al desarrollo del consorcio ITRC-MISTRAL que, liderado por la Universidad de Oxford (UK), está explorando posibles mejoras de la capacidad de las infraestructuras nacionales. En particular, este trabajo se centra en MINERVA, un modelo de infraestructuras de 5G que ayudará a reguladores y operadores a la toma de decisiones a largo plazo.\par

El primer objetivo de este trabajo ha sido la caracterización de las obligaciones de cobertura, analizando el impacto que han supuesto en el desarrollo de la infraestructura de telecomunicaciones en las grandes potencias europeas en el pasado. Además, se ha analizado cómo se comportarán los actores de la subasta de la banda de 700 MHz en función de las posibles situaciones de los operadores y la relevancia que pueda tener la banda en comparación con bandas de frecuencias mayores.\par

Para poder realizar este análisis, se han realizado tres grandes contribuciones al repositorio público de MINERVA. Primero, se ha creado una nueva estrategia, la «700 MHz densification», para comprobar la máxima capacidad que puede ofrecer la banda de 700 MHz. Segundo, se han creado cuatro tipos de obligaciones de cobertura basadas en las impuestas por España, Reino Unido, Alemania y Francia. Tercero, se ha programado un módulo de visualización que presenta los resultados de la simulación en varios formatos para facilitar el análisis del escenario elegido.\par

Finalmente, se han simulado varios escenarios con distintas obligaciones de cobertura y estrategias de aumento de la capacidad de la red para comprobar de dos maneras diferentes el efecto que éstas tienen en los resultados. Primero, determinando el impacto de las obligaciones de cobertura en el ritmo de crecimiento de la red. Segundo, analizando los resultados de simulaciones con variaciones en las bandas disponibles para el operador.\par


\section*{Palabras clave}
Redes 5G, banda de 700 MHz, obligaciones de cobertura, segundo dividendo digital, subastas de espectro.

